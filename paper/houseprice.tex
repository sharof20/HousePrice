\documentclass[11pt, a4paper, leqno]{article}
\usepackage{a4wide}
\usepackage[T1]{fontenc}
\usepackage[utf8]{inputenc}
\usepackage{float, afterpage, rotating, graphicx}
\usepackage{epstopdf}
\usepackage{longtable, booktabs, tabularx}
\usepackage{fancyvrb, moreverb, relsize}
\usepackage{eurosym, calc}
% \usepackage{chngcntr}
\usepackage{amsmath, amssymb, amsfonts, amsthm, bm}
\usepackage{caption}
\usepackage{mdwlist}
\usepackage{xfrac}
\usepackage{setspace}
\usepackage[dvipsnames]{xcolor}
\usepackage{subcaption}
\usepackage{minibox}
% \usepackage{pdf14} % Enable for Manuscriptcentral -- can't handle pdf 1.5
% \usepackage{endfloat} % Enable to move tables / figures to the end. Useful for some
% submissions.

\usepackage[
    natbib=true,
    bibencoding=inputenc,
    bibstyle=authoryear-ibid,
    citestyle=authoryear-comp,
    maxcitenames=3,
    maxbibnames=10,
    useprefix=false,
    sortcites=true,
    backend=biber
]{biblatex}
\AtBeginDocument{\toggletrue{blx@useprefix}}
\AtBeginBibliography{\togglefalse{blx@useprefix}}
\setlength{\bibitemsep}{1.5ex}
\addbibresource{../../paper/refs.bib}

\usepackage[unicode=true]{hyperref}
\hypersetup{
    colorlinks=true,
    linkcolor=black,
    anchorcolor=black,
    citecolor=NavyBlue,
    filecolor=black,
    menucolor=black,
    runcolor=black,
    urlcolor=NavyBlue
}


\widowpenalty=10000
\clubpenalty=10000

\setlength{\parskip}{1ex}
\setlength{\parindent}{0ex}
\setstretch{1.5}


\begin{document}

\title{HousePrice\thanks{Sugarkhuu Radnaa, Juraev Sharofiddin, The University of Bonn. Email: \href{mailto:s6suradn@uni-bonn.de}{\nolinkurl{s6suradn [at] uni-bonn [dot] de}}.}}

\author{Sugarkhuu Radnaa, Juraev Sharofiddin}

\date{
    {\bf Preliminary -- please do not quote}
    \\[1ex]
    \today
}

\maketitle


\begin{abstract}
    Some abstract here.
\end{abstract}

\clearpage


\section{Introduction} % (fold)
\label{sec:introduction}

If you are using this template, please cite this item from the references:
\citet{GaudeckerEconProjectTemplates}.

This project, named HousePrice, aims to predict the prices of residential properties.
The dataset used for this study contains information about 7842 apartments located in Ulaanbaatar, the capital of Mongolia.
The dataset includes features such as price, square footage, location, and other relevant factors.
To ensure the reliability and accuracy of the data, we cleaned and preprocessed it before conducting any analysis.
To gain insights into the dataset, we utilized various data visualization techniques.
These techniques allowed us to identify important trends and patterns in the data, which served as the foundation for our subsequent analysis.
After analyzing the dataset, we constructed a machine learning model capable of accurately predicting the prices of residential properties.
Our model utilizes various features such as location, number of windows, and square footage to generate predictions.
We thoroughly evaluated the performance of our model and selected the best performing one for this study.



\begin{figure}[H]

    \centering
    \includegraphics[width=0.85\textwidth]{../python/figures/smoking_by_marital_status}

    \caption{\emph{Python:} Model predictions of the smoking probability over the
        lifetime. Each colored line represents a case where marital status is fixed to one
        of the values present in the data set.}
    \label{fig:python-predictions}

\end{figure}


\begin{table}[!h]
    \input{../python/tables/estimation_results.tex}
    \caption{\label{tab:python-summary}\emph{Python:} Estimation results of the
        linear Logistic regression.}
\end{table}




% section introduction (end)
\section{Data collection} % (fold)
\label{sec:data_collection}
To gather data for this study, we utilized Selenium and Python to scrape the real estate listing website unegui.mn,
which has been one of the most visited websites in Mongolia since its establishment in 2007.
Selenium is a commonly used tool for web scraping, data extraction, and web-based task automation.
By leveraging Selenium's powerful features, we were able to gain access to the website and extract relevant information.
Our scraping process involved clicking on links, filtering data, and navigating through the website.
We successfully scraped over 140 web pages of unegui.mn, resulting in a dataset of close to 8000 residential properties located in Ulaanbaatar.
The collected data includes features such as price, square footage, year of commissioning, flooring material,
number of balconies, number of garages, window type, door type, district, and construction progress.
To ensure the reliability and accuracy of the data, we implemented a systematic approach to data collection and pre-processing.
This included removing duplicates, cleaning data, and addressing any missing or incomplete data points.
The dataset obtained from our scraping process serves as the foundation for our subsequent data analysis and machine learning model development.

\setstretch{1}
\printbibliography
\setstretch{1.5}


% \appendix

% The chngctr package is needed for the following lines.
% \counterwithin{table}{section}
% \counterwithin{figure}{section}

\end{document}
