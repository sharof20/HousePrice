\documentclass[11pt, a4paper, leqno]{article}
\usepackage{a4wide}
\usepackage[T1]{fontenc}
\usepackage[utf8]{inputenc}
\usepackage{float, afterpage, rotating, graphicx}
\usepackage{epstopdf}
\usepackage{longtable, booktabs, tabularx}
\usepackage{fancyvrb, moreverb, relsize}
\usepackage{eurosym, calc}
% \usepackage{chngcntr}
\usepackage{amsmath, amssymb, amsfonts, amsthm, bm}
\usepackage{caption}
\usepackage{mdwlist}
\usepackage{xfrac}
\usepackage{setspace}
\usepackage[dvipsnames]{xcolor}
\usepackage{subcaption}
\usepackage{minibox}
% \usepackage{pdf14} % Enable for Manuscriptcentral -- can't handle pdf 1.5
% \usepackage{endfloat} % Enable to move tables / figures to the end. Useful for some
% submissions.

\usepackage[
    natbib=true,
    bibencoding=inputenc,
    bibstyle=authoryear-ibid,
    citestyle=authoryear-comp,
    maxcitenames=3,
    maxbibnames=10,
    useprefix=false,
    sortcites=true,
    backend=biber
]{biblatex}
\AtBeginDocument{\toggletrue{blx@useprefix}}
\AtBeginBibliography{\togglefalse{blx@useprefix}}
\setlength{\bibitemsep}{1.5ex}
\addbibresource{../../paper/refs.bib}

\usepackage[unicode=true]{hyperref}
\hypersetup{
    colorlinks=true,
    linkcolor=black,
    anchorcolor=black,
    citecolor=NavyBlue,
    filecolor=black,
    menucolor=black,
    runcolor=black,
    urlcolor=NavyBlue
}


\widowpenalty=10000
\clubpenalty=10000

\setlength{\parskip}{1ex}
\setlength{\parindent}{0ex}
\setstretch{1.5}


\begin{document}

\title{HousePrice\thanks{Sugarkhuu Radnaa, Juraev Sharofiddin, The University of Bonn. Email: \href{mailto:s6suradn@uni-bonn.de}{\nolinkurl{s6suradn [at] uni-bonn [dot] de}}.}}

\author{Sugarkhuu Radnaa, Juraev Sharofiddin}

\date{
    {\bf Preliminary -- please do not quote}
    \\[1ex]
    \today
}

\maketitle


\begin{abstract}
    Some abstract here.
\end{abstract}

\clearpage


\section{Introduction} % (fold)
\label{sec:introduction}

This project, named HousePrice, aims to predict the prices of residential properties.
The dataset used for this study contains information about 7842 apartments located in Ulaanbaatar, the capital of Mongolia.
The dataset includes features such as price, square footage, location, and other relevant factors.
To ensure the reliability and accuracy of the data, we cleaned and preprocessed it before conducting any analysis.
To gain insights into the dataset, we utilized various data visualization techniques.
These techniques allowed us to identify important trends and patterns in the data, which served as the foundation for our subsequent analysis.
After analyzing the dataset, we constructed a machine learning model capable of accurately predicting the prices of residential properties.
Our model utilizes various features such as location, number of windows, and square footage to generate predictions.
We thoroughly evaluated the performance of our model and selected the best performing one for this study.



%\begin{figure}[H]

%    \centering
%    \includegraphics[width=0.85\textwidth]{../python/figures/smoking_by_marital_status}

%    \caption{\emph{Python:} Model predictions of the smoking probability over the
%        lifetime. Each colored line represents a case where marital status is fixed to one
%        of the values present in the data set.}
%    \label{fig:python-predictions}

%\end{figure}


%\begin{table}[!h]
%    \input{../python/tables/estimation_results.tex}
%    \caption{\label{tab:python-summary}\emph{Python:} Estimation results of the
%        linear Logistic regression.}
%\end{table}




% section introduction (end)
\section{Data collection} % (fold)
\label{sec:data_collection}
The aim of our project is to develop a machine learning model that can accurately predict residential property prices in Ulaanbaatar, Mongolia.
To achieve this, we need to identify the key features of residential properties that have an impact on their prices.
Therefore, we opted to collect data from the popular and long-standing \href{https://www.unegui.mn/}{unegui.mn} website, which is widely visited by users seeking information on properties in Ulaanbaatar.
Our decision to use unegui.mn as our data source was based on its extensive reach and reputation as a reliable platform for property listings in the region.\par
The Python code presented in this project's code part describes a web scraping script that collects data on flats in Ulaanbaatar from the "Unegui" real estate website using the Selenium WebDriver,
and stores it in a Pandas dataframe. The script imports the required libraries such as pandas, selenium, and exceptions from the selenium library.
A function named \texttt{run\_collection ()} is defined, which executes the web scraping process.
The function initializes the options for the Chrome browser and the Chrome webdriver, navigates to the website,
and determines the total number of pages that require scraping.
It creates an empty list named "data" to hold the scraped data, then uses a loop to iterate through each page and extract the relevant information.
For each flat, the script clicks on the listing to view its details, retrieves the title, price, and description of the flat, and saves them in a dictionary.
The script then retrieves the flat's attributes, such as the number of windows, square footage, prices, location and door types, and adds them to the same dictionary.
The dictionary is then appended to the data list. The function ultimately returns the data list, which is a list of dictionaries containing the scraped data for each flat.
\section{Data cleaning} % (fold)
\label{sec:data_cleaning}
After the data collection phase, the systematic cleaning of the data begins in order to prepare it for further analysis.
This involves creating small data cleaning functions, which can be called easily and allow for progress tracking.
To ensure clarity and reproducibility, detailed docstrings were added to the functions, specifying their input and output parameters.
As the data is in Mongolian, it needs to be translated to English. A function called \texttt{transliterate\_mn ()} was created for this purpose,
using the translit Python library within the function.
In addition, the data was refined by dropping outliers and other unnecessary data, in order to prepare it for use in building a machine learning model based on the cleaned data.

\setstretch{1}
\printbibliography
\setstretch{1.5}


% \appendix

% The chngctr package is needed for the following lines.
% \counterwithin{table}{section}
% \counterwithin{figure}{section}

\end{document}
