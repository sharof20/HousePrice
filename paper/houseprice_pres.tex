\documentclass[11pt, aspectratio=169]{beamer}
% \documentclass[11pt,handout]{beamer}
\usepackage[T1]{fontenc}
\usepackage[utf8]{inputenc}
\usepackage{textcomp}
\usepackage{float, afterpage, rotating, graphicx}
\usepackage{epstopdf}
\usepackage{longtable, booktabs, tabularx}
\usepackage{fancyvrb, moreverb, relsize}
\usepackage{eurosym, calc}
\usepackage{amsmath, amssymb, amsfonts, amsthm, bm}
\usepackage[
    natbib=true,
    bibencoding=inputenc,
    bibstyle=authoryear-ibid,
    citestyle=authoryear-comp,
    maxcitenames=3,
    maxbibnames=10,
    useprefix=false,
    sortcites=true,
    backend=biber
]{biblatex}
\AtBeginDocument{\toggletrue{blx@useprefix}}
\AtBeginBibliography{\togglefalse{blx@useprefix}}
\setlength{\bibitemsep}{1.5ex}
\addbibresource{refs.bib}

\hypersetup{colorlinks=true, linkcolor=black, anchorcolor=black, citecolor=black, filecolor=black, menucolor=black, runcolor=black, urlcolor=black}

\setbeamertemplate{footline}[frame number]
\setbeamertemplate{navigation symbols}{}
\setbeamertemplate{frametitle}{\centering\vspace{1ex}\insertframetitle\par}


\begin{document}

\title{HousePrice}

\author[Sugarkhuu Radnaa, Juraev Sharofiddin]
{
{\bf Sugarkhuu Radnaa, Juraev Sharofiddin}\\
{\small The University of Bonn}\\[1ex]
}


\begin{frame}
    \titlepage
    \note{~}
\end{frame}


\begin{frame}[t]
    \frametitle{HousePrice: Project Overview}
    \begin{itemize}
        \item Objective: Predict residential property prices in Ulaanbaatar, Mongolia.
        \item Dataset: about 8000 apartments' information including price, square footage, location, and other relevant factors.
        \item Data Analysis: Used data visualization techniques to identify important trends and patterns in the data.
        \item Machine Learning Model: Capable of predicting residential property prices using features such as location, number of windows, square footage, and other factors
    \end{itemize}
    \note{~}
\end{frame}

\begin{frame}[t]
    \frametitle{Key Findings and Implications}
    \begin{itemize}
        \item Key Findings: Square footage is the most important factor influencing residential property prices, followed by location and others.
        \item Property developers can utilize our findings to make informed decisions regarding the location and features of their properties.
        \item Homebuyers can use our model to ensure that they are paying a fair price for a property based on its features.
    \end{itemize}
\end{frame}

\begin{frame}{Data Collection}
    \begin{itemize}
        \item Data Source: unegui.mn, a popular and reliable platform for property listings in the region.
        \item Data Collection: Used a Python web scraping script that collects data on flats in Ulaanbaatar using the Selenium WebDriver.
        \item Libraries: Used pandas, selenium, and exceptions from the selenium library.
        \item Function: Defined a function named \texttt{run\_collection()} to execute the web scraping process and store data in a Pandas dataframe.
        \item Features: The collected data includes information about the price, square footage, location, and other relevant factors of residential properties.
    \end{itemize}
\end{frame}

\begin{frame}{Data Collection}
    \begin{itemize}
        \item Data cleaning functions were created to clean the data systematically and enable progress tracking.
        \item Docstrings were added to the functions to ensure clarity and reproducibility.
        \item The data was refined by dropping outliers and unnecessary data, and translated from Mongolian to English using a Python library.
        \item Additionally, outliers and other unnecessary data were dropped, refining the data for use in building a machine learning model based on the cleaned data.
    \end{itemize}
\end{frame}

\begin{frame}[t]
    \begin{figure}[H]

        \centering
        \includegraphics[width=0.5\textwidth]{../bld/plot/histogram_price.png}

        \caption{\emph{Python:} Frequency Distribution of Prices per Square Meter for a Set of Properties:
        The histogram shows a normal distribution of prices per square meter for a set of properties, with the most common price around 3.
        The x-axis ranges from 0 to 6, and the y-axis represents the frequency or count of properties, ranging from 0 to 500.}
        \label{fig:histogram_price}
    \end{figure}
\end{frame}





% Print black screen only in presentation mode for finishing up.
\mode<beamer> {
    \beamersetaveragebackground{black}
    \begin{frame}
        \frametitle{}
    \end{frame}

    \beamersetaveragebackground{white}
}

\begin{frame}[allowframebreaks]
    \frametitle{References}
    \renewcommand{\bibfont}{\normalfont\footnotesize}
    \printbibliography
\end{frame}

\end{document}
